
% Default to the notebook output style

    


% Inherit from the specified cell style.




    
\documentclass[11pt]{article}

    
    
    \usepackage[T1]{fontenc}
    % Nicer default font (+ math font) than Computer Modern for most use cases
    \usepackage{mathpazo}

    % Basic figure setup, for now with no caption control since it's done
    % automatically by Pandoc (which extracts ![](path) syntax from Markdown).
    \usepackage{graphicx}
    % We will generate all images so they have a width \maxwidth. This means
    % that they will get their normal width if they fit onto the page, but
    % are scaled down if they would overflow the margins.
    \makeatletter
    \def\maxwidth{\ifdim\Gin@nat@width>\linewidth\linewidth
    \else\Gin@nat@width\fi}
    \makeatother
    \let\Oldincludegraphics\includegraphics
    % Set max figure width to be 80% of text width, for now hardcoded.
    \renewcommand{\includegraphics}[1]{\Oldincludegraphics[width=.8\maxwidth]{#1}}
    % Ensure that by default, figures have no caption (until we provide a
    % proper Figure object with a Caption API and a way to capture that
    % in the conversion process - todo).
    \usepackage{caption}
    \DeclareCaptionLabelFormat{nolabel}{}
    \captionsetup{labelformat=nolabel}

    \usepackage{adjustbox} % Used to constrain images to a maximum size 
    \usepackage{xcolor} % Allow colors to be defined
    \usepackage{enumerate} % Needed for markdown enumerations to work
    \usepackage{geometry} % Used to adjust the document margins
    \usepackage{amsmath} % Equations
    \usepackage{amssymb} % Equations
    \usepackage{textcomp} % defines textquotesingle
    % Hack from http://tex.stackexchange.com/a/47451/13684:
    \AtBeginDocument{%
        \def\PYZsq{\textquotesingle}% Upright quotes in Pygmentized code
    }
    \usepackage{upquote} % Upright quotes for verbatim code
    \usepackage{eurosym} % defines \euro
    \usepackage[mathletters]{ucs} % Extended unicode (utf-8) support
    \usepackage[utf8x]{inputenc} % Allow utf-8 characters in the tex document
    \usepackage{fancyvrb} % verbatim replacement that allows latex
    \usepackage{grffile} % extends the file name processing of package graphics 
                         % to support a larger range 
    % The hyperref package gives us a pdf with properly built
    % internal navigation ('pdf bookmarks' for the table of contents,
    % internal cross-reference links, web links for URLs, etc.)
    \usepackage{hyperref}
    \usepackage{longtable} % longtable support required by pandoc >1.10
    \usepackage{booktabs}  % table support for pandoc > 1.12.2
    \usepackage[inline]{enumitem} % IRkernel/repr support (it uses the enumerate* environment)
    \usepackage[normalem]{ulem} % ulem is needed to support strikethroughs (\sout)
                                % normalem makes italics be italics, not underlines
    

    
    
    % Colors for the hyperref package
    \definecolor{urlcolor}{rgb}{0,.145,.698}
    \definecolor{linkcolor}{rgb}{.71,0.21,0.01}
    \definecolor{citecolor}{rgb}{.12,.54,.11}

    % ANSI colors
    \definecolor{ansi-black}{HTML}{3E424D}
    \definecolor{ansi-black-intense}{HTML}{282C36}
    \definecolor{ansi-red}{HTML}{E75C58}
    \definecolor{ansi-red-intense}{HTML}{B22B31}
    \definecolor{ansi-green}{HTML}{00A250}
    \definecolor{ansi-green-intense}{HTML}{007427}
    \definecolor{ansi-yellow}{HTML}{DDB62B}
    \definecolor{ansi-yellow-intense}{HTML}{B27D12}
    \definecolor{ansi-blue}{HTML}{208FFB}
    \definecolor{ansi-blue-intense}{HTML}{0065CA}
    \definecolor{ansi-magenta}{HTML}{D160C4}
    \definecolor{ansi-magenta-intense}{HTML}{A03196}
    \definecolor{ansi-cyan}{HTML}{60C6C8}
    \definecolor{ansi-cyan-intense}{HTML}{258F8F}
    \definecolor{ansi-white}{HTML}{C5C1B4}
    \definecolor{ansi-white-intense}{HTML}{A1A6B2}

    % commands and environments needed by pandoc snippets
    % extracted from the output of `pandoc -s`
    \providecommand{\tightlist}{%
      \setlength{\itemsep}{0pt}\setlength{\parskip}{0pt}}
    \DefineVerbatimEnvironment{Highlighting}{Verbatim}{commandchars=\\\{\}}
    % Add ',fontsize=\small' for more characters per line
    \newenvironment{Shaded}{}{}
    \newcommand{\KeywordTok}[1]{\textcolor[rgb]{0.00,0.44,0.13}{\textbf{{#1}}}}
    \newcommand{\DataTypeTok}[1]{\textcolor[rgb]{0.56,0.13,0.00}{{#1}}}
    \newcommand{\DecValTok}[1]{\textcolor[rgb]{0.25,0.63,0.44}{{#1}}}
    \newcommand{\BaseNTok}[1]{\textcolor[rgb]{0.25,0.63,0.44}{{#1}}}
    \newcommand{\FloatTok}[1]{\textcolor[rgb]{0.25,0.63,0.44}{{#1}}}
    \newcommand{\CharTok}[1]{\textcolor[rgb]{0.25,0.44,0.63}{{#1}}}
    \newcommand{\StringTok}[1]{\textcolor[rgb]{0.25,0.44,0.63}{{#1}}}
    \newcommand{\CommentTok}[1]{\textcolor[rgb]{0.38,0.63,0.69}{\textit{{#1}}}}
    \newcommand{\OtherTok}[1]{\textcolor[rgb]{0.00,0.44,0.13}{{#1}}}
    \newcommand{\AlertTok}[1]{\textcolor[rgb]{1.00,0.00,0.00}{\textbf{{#1}}}}
    \newcommand{\FunctionTok}[1]{\textcolor[rgb]{0.02,0.16,0.49}{{#1}}}
    \newcommand{\RegionMarkerTok}[1]{{#1}}
    \newcommand{\ErrorTok}[1]{\textcolor[rgb]{1.00,0.00,0.00}{\textbf{{#1}}}}
    \newcommand{\NormalTok}[1]{{#1}}
    
    % Additional commands for more recent versions of Pandoc
    \newcommand{\ConstantTok}[1]{\textcolor[rgb]{0.53,0.00,0.00}{{#1}}}
    \newcommand{\SpecialCharTok}[1]{\textcolor[rgb]{0.25,0.44,0.63}{{#1}}}
    \newcommand{\VerbatimStringTok}[1]{\textcolor[rgb]{0.25,0.44,0.63}{{#1}}}
    \newcommand{\SpecialStringTok}[1]{\textcolor[rgb]{0.73,0.40,0.53}{{#1}}}
    \newcommand{\ImportTok}[1]{{#1}}
    \newcommand{\DocumentationTok}[1]{\textcolor[rgb]{0.73,0.13,0.13}{\textit{{#1}}}}
    \newcommand{\AnnotationTok}[1]{\textcolor[rgb]{0.38,0.63,0.69}{\textbf{\textit{{#1}}}}}
    \newcommand{\CommentVarTok}[1]{\textcolor[rgb]{0.38,0.63,0.69}{\textbf{\textit{{#1}}}}}
    \newcommand{\VariableTok}[1]{\textcolor[rgb]{0.10,0.09,0.49}{{#1}}}
    \newcommand{\ControlFlowTok}[1]{\textcolor[rgb]{0.00,0.44,0.13}{\textbf{{#1}}}}
    \newcommand{\OperatorTok}[1]{\textcolor[rgb]{0.40,0.40,0.40}{{#1}}}
    \newcommand{\BuiltInTok}[1]{{#1}}
    \newcommand{\ExtensionTok}[1]{{#1}}
    \newcommand{\PreprocessorTok}[1]{\textcolor[rgb]{0.74,0.48,0.00}{{#1}}}
    \newcommand{\AttributeTok}[1]{\textcolor[rgb]{0.49,0.56,0.16}{{#1}}}
    \newcommand{\InformationTok}[1]{\textcolor[rgb]{0.38,0.63,0.69}{\textbf{\textit{{#1}}}}}
    \newcommand{\WarningTok}[1]{\textcolor[rgb]{0.38,0.63,0.69}{\textbf{\textit{{#1}}}}}
    
    
    % Define a nice break command that doesn't care if a line doesn't already
    % exist.
    \def\br{\hspace*{\fill} \\* }
    % Math Jax compatability definitions
    \def\gt{>}
    \def\lt{<}
    % Document parameters
    \title{Image Operations}
    
    
    

    % Pygments definitions
    
\makeatletter
\def\PY@reset{\let\PY@it=\relax \let\PY@bf=\relax%
    \let\PY@ul=\relax \let\PY@tc=\relax%
    \let\PY@bc=\relax \let\PY@ff=\relax}
\def\PY@tok#1{\csname PY@tok@#1\endcsname}
\def\PY@toks#1+{\ifx\relax#1\empty\else%
    \PY@tok{#1}\expandafter\PY@toks\fi}
\def\PY@do#1{\PY@bc{\PY@tc{\PY@ul{%
    \PY@it{\PY@bf{\PY@ff{#1}}}}}}}
\def\PY#1#2{\PY@reset\PY@toks#1+\relax+\PY@do{#2}}

\expandafter\def\csname PY@tok@w\endcsname{\def\PY@tc##1{\textcolor[rgb]{0.73,0.73,0.73}{##1}}}
\expandafter\def\csname PY@tok@c\endcsname{\let\PY@it=\textit\def\PY@tc##1{\textcolor[rgb]{0.25,0.50,0.50}{##1}}}
\expandafter\def\csname PY@tok@cp\endcsname{\def\PY@tc##1{\textcolor[rgb]{0.74,0.48,0.00}{##1}}}
\expandafter\def\csname PY@tok@k\endcsname{\let\PY@bf=\textbf\def\PY@tc##1{\textcolor[rgb]{0.00,0.50,0.00}{##1}}}
\expandafter\def\csname PY@tok@kp\endcsname{\def\PY@tc##1{\textcolor[rgb]{0.00,0.50,0.00}{##1}}}
\expandafter\def\csname PY@tok@kt\endcsname{\def\PY@tc##1{\textcolor[rgb]{0.69,0.00,0.25}{##1}}}
\expandafter\def\csname PY@tok@o\endcsname{\def\PY@tc##1{\textcolor[rgb]{0.40,0.40,0.40}{##1}}}
\expandafter\def\csname PY@tok@ow\endcsname{\let\PY@bf=\textbf\def\PY@tc##1{\textcolor[rgb]{0.67,0.13,1.00}{##1}}}
\expandafter\def\csname PY@tok@nb\endcsname{\def\PY@tc##1{\textcolor[rgb]{0.00,0.50,0.00}{##1}}}
\expandafter\def\csname PY@tok@nf\endcsname{\def\PY@tc##1{\textcolor[rgb]{0.00,0.00,1.00}{##1}}}
\expandafter\def\csname PY@tok@nc\endcsname{\let\PY@bf=\textbf\def\PY@tc##1{\textcolor[rgb]{0.00,0.00,1.00}{##1}}}
\expandafter\def\csname PY@tok@nn\endcsname{\let\PY@bf=\textbf\def\PY@tc##1{\textcolor[rgb]{0.00,0.00,1.00}{##1}}}
\expandafter\def\csname PY@tok@ne\endcsname{\let\PY@bf=\textbf\def\PY@tc##1{\textcolor[rgb]{0.82,0.25,0.23}{##1}}}
\expandafter\def\csname PY@tok@nv\endcsname{\def\PY@tc##1{\textcolor[rgb]{0.10,0.09,0.49}{##1}}}
\expandafter\def\csname PY@tok@no\endcsname{\def\PY@tc##1{\textcolor[rgb]{0.53,0.00,0.00}{##1}}}
\expandafter\def\csname PY@tok@nl\endcsname{\def\PY@tc##1{\textcolor[rgb]{0.63,0.63,0.00}{##1}}}
\expandafter\def\csname PY@tok@ni\endcsname{\let\PY@bf=\textbf\def\PY@tc##1{\textcolor[rgb]{0.60,0.60,0.60}{##1}}}
\expandafter\def\csname PY@tok@na\endcsname{\def\PY@tc##1{\textcolor[rgb]{0.49,0.56,0.16}{##1}}}
\expandafter\def\csname PY@tok@nt\endcsname{\let\PY@bf=\textbf\def\PY@tc##1{\textcolor[rgb]{0.00,0.50,0.00}{##1}}}
\expandafter\def\csname PY@tok@nd\endcsname{\def\PY@tc##1{\textcolor[rgb]{0.67,0.13,1.00}{##1}}}
\expandafter\def\csname PY@tok@s\endcsname{\def\PY@tc##1{\textcolor[rgb]{0.73,0.13,0.13}{##1}}}
\expandafter\def\csname PY@tok@sd\endcsname{\let\PY@it=\textit\def\PY@tc##1{\textcolor[rgb]{0.73,0.13,0.13}{##1}}}
\expandafter\def\csname PY@tok@si\endcsname{\let\PY@bf=\textbf\def\PY@tc##1{\textcolor[rgb]{0.73,0.40,0.53}{##1}}}
\expandafter\def\csname PY@tok@se\endcsname{\let\PY@bf=\textbf\def\PY@tc##1{\textcolor[rgb]{0.73,0.40,0.13}{##1}}}
\expandafter\def\csname PY@tok@sr\endcsname{\def\PY@tc##1{\textcolor[rgb]{0.73,0.40,0.53}{##1}}}
\expandafter\def\csname PY@tok@ss\endcsname{\def\PY@tc##1{\textcolor[rgb]{0.10,0.09,0.49}{##1}}}
\expandafter\def\csname PY@tok@sx\endcsname{\def\PY@tc##1{\textcolor[rgb]{0.00,0.50,0.00}{##1}}}
\expandafter\def\csname PY@tok@m\endcsname{\def\PY@tc##1{\textcolor[rgb]{0.40,0.40,0.40}{##1}}}
\expandafter\def\csname PY@tok@gh\endcsname{\let\PY@bf=\textbf\def\PY@tc##1{\textcolor[rgb]{0.00,0.00,0.50}{##1}}}
\expandafter\def\csname PY@tok@gu\endcsname{\let\PY@bf=\textbf\def\PY@tc##1{\textcolor[rgb]{0.50,0.00,0.50}{##1}}}
\expandafter\def\csname PY@tok@gd\endcsname{\def\PY@tc##1{\textcolor[rgb]{0.63,0.00,0.00}{##1}}}
\expandafter\def\csname PY@tok@gi\endcsname{\def\PY@tc##1{\textcolor[rgb]{0.00,0.63,0.00}{##1}}}
\expandafter\def\csname PY@tok@gr\endcsname{\def\PY@tc##1{\textcolor[rgb]{1.00,0.00,0.00}{##1}}}
\expandafter\def\csname PY@tok@ge\endcsname{\let\PY@it=\textit}
\expandafter\def\csname PY@tok@gs\endcsname{\let\PY@bf=\textbf}
\expandafter\def\csname PY@tok@gp\endcsname{\let\PY@bf=\textbf\def\PY@tc##1{\textcolor[rgb]{0.00,0.00,0.50}{##1}}}
\expandafter\def\csname PY@tok@go\endcsname{\def\PY@tc##1{\textcolor[rgb]{0.53,0.53,0.53}{##1}}}
\expandafter\def\csname PY@tok@gt\endcsname{\def\PY@tc##1{\textcolor[rgb]{0.00,0.27,0.87}{##1}}}
\expandafter\def\csname PY@tok@err\endcsname{\def\PY@bc##1{\setlength{\fboxsep}{0pt}\fcolorbox[rgb]{1.00,0.00,0.00}{1,1,1}{\strut ##1}}}
\expandafter\def\csname PY@tok@kc\endcsname{\let\PY@bf=\textbf\def\PY@tc##1{\textcolor[rgb]{0.00,0.50,0.00}{##1}}}
\expandafter\def\csname PY@tok@kd\endcsname{\let\PY@bf=\textbf\def\PY@tc##1{\textcolor[rgb]{0.00,0.50,0.00}{##1}}}
\expandafter\def\csname PY@tok@kn\endcsname{\let\PY@bf=\textbf\def\PY@tc##1{\textcolor[rgb]{0.00,0.50,0.00}{##1}}}
\expandafter\def\csname PY@tok@kr\endcsname{\let\PY@bf=\textbf\def\PY@tc##1{\textcolor[rgb]{0.00,0.50,0.00}{##1}}}
\expandafter\def\csname PY@tok@bp\endcsname{\def\PY@tc##1{\textcolor[rgb]{0.00,0.50,0.00}{##1}}}
\expandafter\def\csname PY@tok@fm\endcsname{\def\PY@tc##1{\textcolor[rgb]{0.00,0.00,1.00}{##1}}}
\expandafter\def\csname PY@tok@vc\endcsname{\def\PY@tc##1{\textcolor[rgb]{0.10,0.09,0.49}{##1}}}
\expandafter\def\csname PY@tok@vg\endcsname{\def\PY@tc##1{\textcolor[rgb]{0.10,0.09,0.49}{##1}}}
\expandafter\def\csname PY@tok@vi\endcsname{\def\PY@tc##1{\textcolor[rgb]{0.10,0.09,0.49}{##1}}}
\expandafter\def\csname PY@tok@vm\endcsname{\def\PY@tc##1{\textcolor[rgb]{0.10,0.09,0.49}{##1}}}
\expandafter\def\csname PY@tok@sa\endcsname{\def\PY@tc##1{\textcolor[rgb]{0.73,0.13,0.13}{##1}}}
\expandafter\def\csname PY@tok@sb\endcsname{\def\PY@tc##1{\textcolor[rgb]{0.73,0.13,0.13}{##1}}}
\expandafter\def\csname PY@tok@sc\endcsname{\def\PY@tc##1{\textcolor[rgb]{0.73,0.13,0.13}{##1}}}
\expandafter\def\csname PY@tok@dl\endcsname{\def\PY@tc##1{\textcolor[rgb]{0.73,0.13,0.13}{##1}}}
\expandafter\def\csname PY@tok@s2\endcsname{\def\PY@tc##1{\textcolor[rgb]{0.73,0.13,0.13}{##1}}}
\expandafter\def\csname PY@tok@sh\endcsname{\def\PY@tc##1{\textcolor[rgb]{0.73,0.13,0.13}{##1}}}
\expandafter\def\csname PY@tok@s1\endcsname{\def\PY@tc##1{\textcolor[rgb]{0.73,0.13,0.13}{##1}}}
\expandafter\def\csname PY@tok@mb\endcsname{\def\PY@tc##1{\textcolor[rgb]{0.40,0.40,0.40}{##1}}}
\expandafter\def\csname PY@tok@mf\endcsname{\def\PY@tc##1{\textcolor[rgb]{0.40,0.40,0.40}{##1}}}
\expandafter\def\csname PY@tok@mh\endcsname{\def\PY@tc##1{\textcolor[rgb]{0.40,0.40,0.40}{##1}}}
\expandafter\def\csname PY@tok@mi\endcsname{\def\PY@tc##1{\textcolor[rgb]{0.40,0.40,0.40}{##1}}}
\expandafter\def\csname PY@tok@il\endcsname{\def\PY@tc##1{\textcolor[rgb]{0.40,0.40,0.40}{##1}}}
\expandafter\def\csname PY@tok@mo\endcsname{\def\PY@tc##1{\textcolor[rgb]{0.40,0.40,0.40}{##1}}}
\expandafter\def\csname PY@tok@ch\endcsname{\let\PY@it=\textit\def\PY@tc##1{\textcolor[rgb]{0.25,0.50,0.50}{##1}}}
\expandafter\def\csname PY@tok@cm\endcsname{\let\PY@it=\textit\def\PY@tc##1{\textcolor[rgb]{0.25,0.50,0.50}{##1}}}
\expandafter\def\csname PY@tok@cpf\endcsname{\let\PY@it=\textit\def\PY@tc##1{\textcolor[rgb]{0.25,0.50,0.50}{##1}}}
\expandafter\def\csname PY@tok@c1\endcsname{\let\PY@it=\textit\def\PY@tc##1{\textcolor[rgb]{0.25,0.50,0.50}{##1}}}
\expandafter\def\csname PY@tok@cs\endcsname{\let\PY@it=\textit\def\PY@tc##1{\textcolor[rgb]{0.25,0.50,0.50}{##1}}}

\def\PYZbs{\char`\\}
\def\PYZus{\char`\_}
\def\PYZob{\char`\{}
\def\PYZcb{\char`\}}
\def\PYZca{\char`\^}
\def\PYZam{\char`\&}
\def\PYZlt{\char`\<}
\def\PYZgt{\char`\>}
\def\PYZsh{\char`\#}
\def\PYZpc{\char`\%}
\def\PYZdl{\char`\$}
\def\PYZhy{\char`\-}
\def\PYZsq{\char`\'}
\def\PYZdq{\char`\"}
\def\PYZti{\char`\~}
% for compatibility with earlier versions
\def\PYZat{@}
\def\PYZlb{[}
\def\PYZrb{]}
\makeatother


    % Exact colors from NB
    \definecolor{incolor}{rgb}{0.0, 0.0, 0.5}
    \definecolor{outcolor}{rgb}{0.545, 0.0, 0.0}



    
    % Prevent overflowing lines due to hard-to-break entities
    \sloppy 
    % Setup hyperref package
    \hypersetup{
      breaklinks=true,  % so long urls are correctly broken across lines
      colorlinks=true,
      urlcolor=urlcolor,
      linkcolor=linkcolor,
      citecolor=citecolor,
      }
    % Slightly bigger margins than the latex defaults
    
    \geometry{verbose,tmargin=1in,bmargin=1in,lmargin=1in,rmargin=1in}
    
    

    \begin{document}
    
    
    \maketitle
    
    

    
    \section{Image operations}\label{image-operations}

    Table of Contents{}

{Image operations}

{1.0 Import Libraries}

{1.1 Read the Image}

{1.2 Convert whole Image to Grayscale}

{1.3 Convert Range of Image to another color}

{1.4 Images Geometric Transformations}

{1.4.1 Image Scaling}

{1.4.2 Image rotation}

{1.5 Image Thresholding}

{1.5.1 simple thresholding}

{1.5.2 adaptive thresholding}

{1.6 Image Arithmetics}

{1.6.1 Image Addition}

{1.6.1.1 absolute addition}

{1.6.1.2 wieghted addition}

    In image manipulations, we can perform operations on colors as well as
drowing and adding fonts on them. The dataset we have is grayscaled. I
will get a colorful image so that effects of colors will be clear.

    \subsection{1.0 Import Libraries}\label{import-libraries}

    \begin{Verbatim}[commandchars=\\\{\}]
{\color{incolor}In [{\color{incolor}1}]:} \PY{c+c1}{\PYZsh{} important imports:}
        \PY{k+kn}{import} \PY{n+nn}{cv2}
        \PY{k+kn}{import} \PY{n+nn}{numpy} \PY{k}{as} \PY{n+nn}{np}
        \PY{k+kn}{import} \PY{n+nn}{matplotlib}\PY{n+nn}{.}\PY{n+nn}{pyplot} \PY{k}{as} \PY{n+nn}{plt}
        
        \PY{c+c1}{\PYZsh{}\PYZsh{} make sure to install dependencies above.}
\end{Verbatim}


    \subsection{1.1 Read the Image}\label{read-the-image}

    \begin{Verbatim}[commandchars=\\\{\}]
{\color{incolor}In [{\color{incolor}2}]:} \PY{c+c1}{\PYZsh{} reading the image}
        \PY{n}{clrd\PYZus{}img} \PY{o}{=} \PY{n}{cv2}\PY{o}{.}\PY{n}{imread}\PY{p}{(}\PY{l+s+s1}{\PYZsq{}}\PY{l+s+s1}{colorful\PYZus{}image.jpeg}\PY{l+s+s1}{\PYZsq{}}\PY{p}{,} \PY{n}{cv2}\PY{o}{.}\PY{n}{IMREAD\PYZus{}COLOR}\PY{p}{)}
\end{Verbatim}


    \begin{Verbatim}[commandchars=\\\{\}]
{\color{incolor}In [{\color{incolor}3}]:} \PY{c+c1}{\PYZsh{} view the image with plot}
        \PY{n}{plt}\PY{o}{.}\PY{n}{imshow}\PY{p}{(}\PY{n}{clrd\PYZus{}img}\PY{p}{)}
\end{Verbatim}


\begin{Verbatim}[commandchars=\\\{\}]
{\color{outcolor}Out[{\color{outcolor}3}]:} <matplotlib.image.AxesImage at 0x7f50861cb278>
\end{Verbatim}
            
    \begin{center}
    \adjustimage{max size={0.9\linewidth}{0.9\paperheight}}{output_7_1.png}
    \end{center}
    { \hspace*{\fill} \\}
    
    \subsection{1.2 Convert whole Image to
Grayscale}\label{convert-whole-image-to-grayscale}

    \begin{Verbatim}[commandchars=\\\{\}]
{\color{incolor}In [{\color{incolor}4}]:} \PY{c+c1}{\PYZsh{}change the image to grayscale}
        \PY{n}{gray\PYZus{}crld\PYZus{}img} \PY{o}{=} \PY{n}{cv2}\PY{o}{.}\PY{n}{cvtColor}\PY{p}{(}\PY{n}{clrd\PYZus{}img}\PY{p}{,} \PY{n}{cv2}\PY{o}{.}\PY{n}{COLOR\PYZus{}BGR2GRAY}\PY{p}{)}
        \PY{n}{plt}\PY{o}{.}\PY{n}{imshow}\PY{p}{(}\PY{n}{gray\PYZus{}crld\PYZus{}img}\PY{p}{)}
\end{Verbatim}


\begin{Verbatim}[commandchars=\\\{\}]
{\color{outcolor}Out[{\color{outcolor}4}]:} <matplotlib.image.AxesImage at 0x7f508616d438>
\end{Verbatim}
            
    \begin{center}
    \adjustimage{max size={0.9\linewidth}{0.9\paperheight}}{output_9_1.png}
    \end{center}
    { \hspace*{\fill} \\}
    
    \subsection{1.3 Convert Range of Image to another
color}\label{convert-range-of-image-to-another-color}

    \begin{Verbatim}[commandchars=\\\{\}]
{\color{incolor}In [{\color{incolor}5}]:} \PY{c+c1}{\PYZsh{} you can use numpy indecies to take a region of image}
        \PY{c+c1}{\PYZsh{} see to above pic, to take a region of it:}
        \PY{c+c1}{\PYZsh{}the hieght interval is specified in the first index, }
        \PY{c+c1}{\PYZsh{} and the width interval is specified in the second index}
        \PY{n}{roi} \PY{o}{=} \PY{n}{clrd\PYZus{}img}\PY{p}{[}\PY{l+m+mi}{40}\PY{p}{:}\PY{l+m+mi}{120}\PY{p}{,} \PY{l+m+mi}{160}\PY{p}{:}\PY{l+m+mi}{290}\PY{p}{]}
        
        \PY{c+c1}{\PYZsh{} you can also choose a specific pixel to change its value as follows:}
        \PY{c+c1}{\PYZsh{} px = img[100, 100]}
        \PY{n}{plt}\PY{o}{.}\PY{n}{imshow}\PY{p}{(}\PY{n}{roi}\PY{p}{)}
\end{Verbatim}


\begin{Verbatim}[commandchars=\\\{\}]
{\color{outcolor}Out[{\color{outcolor}5}]:} <matplotlib.image.AxesImage at 0x7f504e73f630>
\end{Verbatim}
            
    \begin{center}
    \adjustimage{max size={0.9\linewidth}{0.9\paperheight}}{output_11_1.png}
    \end{center}
    { \hspace*{\fill} \\}
    
    \begin{Verbatim}[commandchars=\\\{\}]
{\color{incolor}In [{\color{incolor}6}]:} \PY{n}{gray\PYZus{}crld\PYZus{}roi} \PY{o}{=} \PY{n}{cv2}\PY{o}{.}\PY{n}{cvtColor}\PY{p}{(}\PY{n}{roi}\PY{p}{,} \PY{n}{cv2}\PY{o}{.}\PY{n}{COLOR\PYZus{}BGR2GRAY}\PY{p}{)}
        \PY{n}{plt}\PY{o}{.}\PY{n}{imshow}\PY{p}{(}\PY{n}{gray\PYZus{}crld\PYZus{}roi}\PY{p}{)} 
\end{Verbatim}


\begin{Verbatim}[commandchars=\\\{\}]
{\color{outcolor}Out[{\color{outcolor}6}]:} <matplotlib.image.AxesImage at 0x7f504e69ce80>
\end{Verbatim}
            
    \begin{center}
    \adjustimage{max size={0.9\linewidth}{0.9\paperheight}}{output_12_1.png}
    \end{center}
    { \hspace*{\fill} \\}
    
    You can also change this part to any color you like. for example to
\emph{red}.

    \begin{Verbatim}[commandchars=\\\{\}]
{\color{incolor}In [{\color{incolor}7}]:} \PY{n}{clrd\PYZus{}img\PYZus{}with\PYZus{}red} \PY{o}{=} \PY{n}{clrd\PYZus{}img}\PY{o}{.}\PY{n}{copy}\PY{p}{(}\PY{p}{)}
        \PY{n}{clrd\PYZus{}img\PYZus{}with\PYZus{}red}\PY{p}{[}\PY{l+m+mi}{40}\PY{p}{:}\PY{l+m+mi}{120}\PY{p}{,} \PY{l+m+mi}{160}\PY{p}{:}\PY{l+m+mi}{290}\PY{p}{]} \PY{o}{=} \PY{p}{[}\PY{l+m+mi}{255}\PY{p}{,} \PY{l+m+mi}{0}\PY{p}{,} \PY{l+m+mi}{0}\PY{p}{]}
        \PY{n}{plt}\PY{o}{.}\PY{n}{imshow}\PY{p}{(}\PY{n}{clrd\PYZus{}img\PYZus{}with\PYZus{}red}\PY{p}{)}
\end{Verbatim}


\begin{Verbatim}[commandchars=\\\{\}]
{\color{outcolor}Out[{\color{outcolor}7}]:} <matplotlib.image.AxesImage at 0x7f504e687a20>
\end{Verbatim}
            
    \begin{center}
    \adjustimage{max size={0.9\linewidth}{0.9\paperheight}}{output_14_1.png}
    \end{center}
    { \hspace*{\fill} \\}
    
    you can move parts of image to other locations as well. look to the
following code:

    \begin{Verbatim}[commandchars=\\\{\}]
{\color{incolor}In [{\color{incolor}8}]:} \PY{n}{plt}\PY{o}{.}\PY{n}{imshow}\PY{p}{(}\PY{n}{clrd\PYZus{}img}\PY{p}{)}
\end{Verbatim}


\begin{Verbatim}[commandchars=\\\{\}]
{\color{outcolor}Out[{\color{outcolor}8}]:} <matplotlib.image.AxesImage at 0x7f504e5e9ef0>
\end{Verbatim}
            
    \begin{center}
    \adjustimage{max size={0.9\linewidth}{0.9\paperheight}}{output_16_1.png}
    \end{center}
    { \hspace*{\fill} \\}
    
    \begin{Verbatim}[commandchars=\\\{\}]
{\color{incolor}In [{\color{incolor}9}]:} \PY{c+c1}{\PYZsh{} keep in mind that the size of each region should be identical.}
        \PY{c+c1}{\PYZsh{} that is img[a:b, c:d] = img[a\PYZsq{}:b\PYZsq{}, c\PYZsq{}:d\PYZsq{}] where:}
        \PY{c+c1}{\PYZsh{} |a\PYZhy{}b| = |a\PYZsq{}\PYZhy{}b\PYZsq{}| and |c\PYZhy{}d| = |c\PYZsq{}\PYZhy{}d\PYZsq{}|}
        \PY{n}{clrd\PYZus{}img\PYZus{}duplicated\PYZus{}parts} \PY{o}{=} \PY{n}{clrd\PYZus{}img}\PY{o}{.}\PY{n}{copy}\PY{p}{(}\PY{p}{)}
        \PY{n}{clrd\PYZus{}img\PYZus{}duplicated\PYZus{}parts}\PY{p}{[}\PY{l+m+mi}{120}\PY{p}{:}\PY{l+m+mi}{170}\PY{p}{,} \PY{l+m+mi}{150}\PY{p}{:}\PY{l+m+mi}{270}\PY{p}{]} \PY{o}{=} \PY{n}{clrd\PYZus{}img}\PY{p}{[}\PY{l+m+mi}{40}\PY{p}{:}\PY{l+m+mi}{90}\PY{p}{,} \PY{l+m+mi}{160}\PY{p}{:}\PY{l+m+mi}{280}\PY{p}{]}
        \PY{n}{plt}\PY{o}{.}\PY{n}{imshow}\PY{p}{(}\PY{n}{clrd\PYZus{}img\PYZus{}duplicated\PYZus{}parts}\PY{p}{)}
\end{Verbatim}


\begin{Verbatim}[commandchars=\\\{\}]
{\color{outcolor}Out[{\color{outcolor}9}]:} <matplotlib.image.AxesImage at 0x7f504e5d3c88>
\end{Verbatim}
            
    \begin{center}
    \adjustimage{max size={0.9\linewidth}{0.9\paperheight}}{output_17_1.png}
    \end{center}
    { \hspace*{\fill} \\}
    
    \subsection{1.4 Images Geometric
Transformations}\label{images-geometric-transformations}

    \subsubsection{1.4.1 Image Scaling}\label{image-scaling}

    OpenCV comes with the method cv2.resize() to resizes images. This
resizing includes scaling the image up or down as desired.

    \begin{Verbatim}[commandchars=\\\{\}]
{\color{incolor}In [{\color{incolor}10}]:} \PY{n}{clrd\PYZus{}img\PYZus{}resized} \PY{o}{=} \PY{n}{clrd\PYZus{}img}\PY{o}{.}\PY{n}{copy}\PY{p}{(}\PY{p}{)}
         \PY{n}{clrd\PYZus{}img\PYZus{}resized} \PY{o}{=} \PY{n}{cv2}\PY{o}{.}\PY{n}{resize}\PY{p}{(}\PY{n}{clrd\PYZus{}img\PYZus{}resized}\PY{p}{,}\PY{p}{(}\PY{l+m+mi}{300}\PY{p}{,}\PY{l+m+mi}{180}\PY{p}{)}\PY{p}{,} \PY{n}{fx}\PY{o}{=}\PY{l+m+mi}{2}\PY{p}{,} \PY{n}{fy}\PY{o}{=}\PY{l+m+mi}{2}\PY{p}{)}
         \PY{c+c1}{\PYZsh{} the first parameter is the image.}
         \PY{c+c1}{\PYZsh{} the second paramter is used to define the length of the axis to show}
         \PY{c+c1}{\PYZsh{}       if it is the same as the original image, no scaling will be done,}
         \PY{c+c1}{\PYZsh{}       if it is (0,0), the image will be shown as default with doubled axis scales}
         \PY{c+c1}{\PYZsh{}       you can change the value to see the effect.}
         \PY{c+c1}{\PYZsh{} the third parameter is the scaling regarding the x\PYZhy{}axis}
         \PY{c+c1}{\PYZsh{} the fourth paramter is the scaling regarding the y\PYZhy{}axis}
         \PY{n}{plt}\PY{o}{.}\PY{n}{imshow}\PY{p}{(}\PY{n}{clrd\PYZus{}img\PYZus{}resized}\PY{p}{)}
\end{Verbatim}


\begin{Verbatim}[commandchars=\\\{\}]
{\color{outcolor}Out[{\color{outcolor}10}]:} <matplotlib.image.AxesImage at 0x7f504e53a550>
\end{Verbatim}
            
    \begin{center}
    \adjustimage{max size={0.9\linewidth}{0.9\paperheight}}{output_21_1.png}
    \end{center}
    { \hspace*{\fill} \\}
    
    \subsubsection{1.4.2 Image rotation}\label{image-rotation}

    Image rotation is a little bit tricky. To do it, you must first find the
rotation matrix then feed it to a function called warpAffine(). These
functions are all in the OpenCV

    \begin{Verbatim}[commandchars=\\\{\}]
{\color{incolor}In [{\color{incolor}11}]:} \PY{n}{clrd\PYZus{}img\PYZus{}rotated} \PY{o}{=} \PY{n}{clrd\PYZus{}img}\PY{o}{.}\PY{n}{copy}\PY{p}{(}\PY{p}{)}
         \PY{n}{rows}\PY{p}{,}\PY{n}{cols} \PY{o}{=} \PY{n}{clrd\PYZus{}img\PYZus{}rotated}\PY{o}{.}\PY{n}{shape}\PY{p}{[}\PY{l+m+mi}{0}\PY{p}{]}\PY{p}{,} \PY{n}{clrd\PYZus{}img\PYZus{}rotated}\PY{o}{.}\PY{n}{shape}\PY{p}{[}\PY{l+m+mi}{1}\PY{p}{]}
         
         \PY{n}{angle} \PY{o}{=} \PY{l+m+mi}{90}
         \PY{n}{M} \PY{o}{=} \PY{n}{cv2}\PY{o}{.}\PY{n}{getRotationMatrix2D}\PY{p}{(}\PY{p}{(}\PY{n}{cols}\PY{o}{/}\PY{l+m+mi}{2}\PY{p}{,} \PY{n}{rows}\PY{o}{/}\PY{l+m+mi}{2}\PY{p}{)}\PY{p}{,} \PY{n}{angle}\PY{p}{,} \PY{l+m+mi}{1}\PY{p}{)}
         \PY{n}{clrd\PYZus{}img\PYZus{}rotated} \PY{o}{=} \PY{n}{cv2}\PY{o}{.}\PY{n}{warpAffine}\PY{p}{(}\PY{n}{clrd\PYZus{}img\PYZus{}rotated}\PY{p}{,} \PY{n}{M}\PY{p}{,} \PY{p}{(}\PY{n}{cols}\PY{p}{,} \PY{n}{rows}\PY{p}{)}\PY{p}{)}
         \PY{n}{plt}\PY{o}{.}\PY{n}{imshow}\PY{p}{(}\PY{n}{clrd\PYZus{}img\PYZus{}rotated}\PY{p}{)}
\end{Verbatim}


\begin{Verbatim}[commandchars=\\\{\}]
{\color{outcolor}Out[{\color{outcolor}11}]:} <matplotlib.image.AxesImage at 0x7f504e4a2748>
\end{Verbatim}
            
    \begin{center}
    \adjustimage{max size={0.9\linewidth}{0.9\paperheight}}{output_24_1.png}
    \end{center}
    { \hspace*{\fill} \\}
    
    To be honest, I only get to know how to change the angle, but other
parameters are not even clear in the documentation. I may read more
about it later.

    \subsection{1.5 Image Thresholding}\label{image-thresholding}

    \subsubsection{1.5.1 simple thresholding}\label{simple-thresholding}

    Image Thresholding is a very important operation. It is used, I guess
and you will probably agree with me after while, by many commercial
applications such as \href{https://www.camscanner.com/}{CamScanner}.

The idea is very simple. For a given image, Image thresholding is to
change all pixels of the image having value greater that threshold to a
given fixed value. The same thing is applied to the pixels below this
threshold.

We may apply this on the above colorful images but they will not depict
its crucial use. I will apply it on one of them and apply it, then, on
another image to show it importance.

\textbf{It is important to know that this operation is applied to a gray
scaled images} to know why, values of gray scaled images are of type
numpy arrays to represent the values of RGB, but pixel values of gray
scaled images are just numbers of black intensity between 0,255
inclusive.

    \begin{Verbatim}[commandchars=\\\{\}]
{\color{incolor}In [{\color{incolor}12}]:} \PY{n}{clrd\PYZus{}img\PYZus{}thresholded} \PY{o}{=} \PY{n}{clrd\PYZus{}img}\PY{o}{.}\PY{n}{copy}\PY{p}{(}\PY{p}{)}
         \PY{n}{clrd\PYZus{}img\PYZus{}thresholded} \PY{o}{=} \PY{n}{cv2}\PY{o}{.}\PY{n}{cvtColor}\PY{p}{(}\PY{n}{clrd\PYZus{}img\PYZus{}thresholded}\PY{p}{,} \PY{n}{cv2}\PY{o}{.}\PY{n}{COLOR\PYZus{}BGR2GRAY}\PY{p}{)}
         \PY{n}{ret}\PY{p}{,} \PY{n}{threshold1} \PY{o}{=} \PY{n}{cv2}\PY{o}{.}\PY{n}{threshold}\PY{p}{(}\PY{n}{clrd\PYZus{}img\PYZus{}thresholded}\PY{p}{,} \PY{l+m+mi}{125}\PY{p}{,} \PY{l+m+mi}{50}\PY{p}{,} \PY{n}{cv2}\PY{o}{.}\PY{n}{THRESH\PYZus{}BINARY}\PY{p}{)}
         \PY{c+c1}{\PYZsh{} first parameter is the image to be thresholded.}
         \PY{c+c1}{\PYZsh{} second paramter is the threshold by which pixels are calssified.}
         \PY{c+c1}{\PYZsh{} third parameter is the maxVal to which pixels exceeding the threshold are converted.}
         \PY{c+c1}{\PYZsh{} fourth paramter is the threshold type. Better to take a look at their differences in this page: }
         \PY{c+c1}{\PYZsh{} https://docs.opencv.org/2.4/doc/tutorials/imgproc/threshold/threshold.html}
\end{Verbatim}


    \begin{Verbatim}[commandchars=\\\{\}]
{\color{incolor}In [{\color{incolor}13}]:} \PY{n+nb}{print}\PY{p}{(}\PY{l+s+s1}{\PYZsq{}}\PY{l+s+s1}{original image}\PY{l+s+s1}{\PYZsq{}}\PY{p}{)}
         \PY{n}{plt}\PY{o}{.}\PY{n}{imshow}\PY{p}{(}\PY{n}{clrd\PYZus{}img\PYZus{}thresholded}\PY{p}{)}
\end{Verbatim}


    \begin{Verbatim}[commandchars=\\\{\}]
original image

    \end{Verbatim}

\begin{Verbatim}[commandchars=\\\{\}]
{\color{outcolor}Out[{\color{outcolor}13}]:} <matplotlib.image.AxesImage at 0x7f504e4854a8>
\end{Verbatim}
            
    \begin{center}
    \adjustimage{max size={0.9\linewidth}{0.9\paperheight}}{output_30_2.png}
    \end{center}
    { \hspace*{\fill} \\}
    
    \begin{Verbatim}[commandchars=\\\{\}]
{\color{incolor}In [{\color{incolor}14}]:} \PY{n+nb}{print}\PY{p}{(}\PY{l+s+s1}{\PYZsq{}}\PY{l+s+s1}{thresholded image}\PY{l+s+s1}{\PYZsq{}}\PY{p}{)}
         \PY{n}{plt}\PY{o}{.}\PY{n}{imshow}\PY{p}{(}\PY{n}{threshold1}\PY{p}{,} \PY{n}{cmap}\PY{o}{=}\PY{l+s+s1}{\PYZsq{}}\PY{l+s+s1}{gray}\PY{l+s+s1}{\PYZsq{}}\PY{p}{)}
\end{Verbatim}


    \begin{Verbatim}[commandchars=\\\{\}]
thresholded image

    \end{Verbatim}

\begin{Verbatim}[commandchars=\\\{\}]
{\color{outcolor}Out[{\color{outcolor}14}]:} <matplotlib.image.AxesImage at 0x7f504e3e4e80>
\end{Verbatim}
            
    \begin{center}
    \adjustimage{max size={0.9\linewidth}{0.9\paperheight}}{output_31_2.png}
    \end{center}
    { \hspace*{\fill} \\}
    
    Matplotlib Library is not good at showing grayscaled images. Actually,
it shows all white scales as only one color. the White color.Better to
move to cv2.imshow() in the following code:

    \begin{Verbatim}[commandchars=\\\{\}]
{\color{incolor}In [{\color{incolor}15}]:} \PY{n}{cv2}\PY{o}{.}\PY{n}{imshow}\PY{p}{(}\PY{l+s+s1}{\PYZsq{}}\PY{l+s+s1}{test}\PY{l+s+s1}{\PYZsq{}}\PY{p}{,}\PY{n}{threshold1}\PY{p}{)}
         \PY{n}{cv2}\PY{o}{.}\PY{n}{waitKey}\PY{p}{(}\PY{l+m+mi}{0}\PY{p}{)}
         \PY{n}{cv2}\PY{o}{.}\PY{n}{destroyAllWindows}\PY{p}{(}\PY{p}{)}
\end{Verbatim}


    To give a better example showing how useful is this technique, I will
upload an image of a textual paper. The paper is not clear. See it
below:

    \begin{Verbatim}[commandchars=\\\{\}]
{\color{incolor}In [{\color{incolor}18}]:} \PY{n}{paper\PYZus{}img} \PY{o}{=} \PY{n}{cv2}\PY{o}{.}\PY{n}{imread}\PY{p}{(}\PY{l+s+s1}{\PYZsq{}}\PY{l+s+s1}{paper\PYZus{}text.jpg}\PY{l+s+s1}{\PYZsq{}}\PY{p}{)}
         \PY{n}{plt}\PY{o}{.}\PY{n}{imshow}\PY{p}{(}\PY{n}{paper\PYZus{}img}\PY{p}{)}
\end{Verbatim}


\begin{Verbatim}[commandchars=\\\{\}]
{\color{outcolor}Out[{\color{outcolor}18}]:} <matplotlib.image.AxesImage at 0x7f504d52f5c0>
\end{Verbatim}
            
    \begin{center}
    \adjustimage{max size={0.9\linewidth}{0.9\paperheight}}{output_35_1.png}
    \end{center}
    { \hspace*{\fill} \\}
    
    Let us apply thresholding to this image.

since the image is not clear, a low threshold should be okay to
distinguish textual pixels from white ones.

    \begin{Verbatim}[commandchars=\\\{\}]
{\color{incolor}In [{\color{incolor}19}]:} \PY{n}{paper\PYZus{}img\PYZus{}grayscale} \PY{o}{=} \PY{n}{cv2}\PY{o}{.}\PY{n}{cvtColor}\PY{p}{(}\PY{n}{paper\PYZus{}img}\PY{p}{,} \PY{n}{cv2}\PY{o}{.}\PY{n}{COLOR\PYZus{}BGR2GRAY}\PY{p}{)}
         \PY{n}{ret}\PY{p}{,} \PY{n}{thresholded2} \PY{o}{=} \PY{n}{cv2}\PY{o}{.}\PY{n}{cv2}\PY{o}{.}\PY{n}{threshold}\PY{p}{(}\PY{n}{paper\PYZus{}img\PYZus{}grayscale}\PY{p}{,} \PY{l+m+mi}{30}\PY{p}{,} \PY{l+m+mi}{255}\PY{p}{,} \PY{n}{cv2}\PY{o}{.}\PY{n}{THRESH\PYZus{}BINARY}\PY{p}{)}
\end{Verbatim}


    \begin{Verbatim}[commandchars=\\\{\}]
{\color{incolor}In [{\color{incolor}21}]:} \PY{n}{plt}\PY{o}{.}\PY{n}{imshow}\PY{p}{(}\PY{n}{thresholded2}\PY{p}{,} \PY{n}{cmap}\PY{o}{=}\PY{l+s+s1}{\PYZsq{}}\PY{l+s+s1}{gray}\PY{l+s+s1}{\PYZsq{}}\PY{p}{)}
\end{Verbatim}


\begin{Verbatim}[commandchars=\\\{\}]
{\color{outcolor}Out[{\color{outcolor}21}]:} <matplotlib.image.AxesImage at 0x7f504d30bcc0>
\end{Verbatim}
            
    \begin{center}
    \adjustimage{max size={0.9\linewidth}{0.9\paperheight}}{output_38_1.png}
    \end{center}
    { \hspace*{\fill} \\}
    
    as you can see, The image is converted to black and white image where
textual pixels are separated. However, there are some regions of the
image where black color is destructing the text. In this case, we will
use another type of thresholding explained in the next section.

    \subsubsection{1.5.2 adaptive thresholding}\label{adaptive-thresholding}

    In adaptive thresholding, the algorithm check for smaller regions so
that the result is optimized. In that case, the algorithm finds
different thresholds for different regions of the image depending on the
lightening of the area and other conditions. As its counterpart, it has
two types: - cv2.ADAPTIVE\_THRESH\_MEAN\_C -
cv2.ADAPTIVE\_THRESH\_GAUSSIAN\_C

The first type takes the mean of the selected region to be the threshold
while the threshold in the second type is the weighted sum of pixels
values in the selected area and weights are a gaussian window.

    \begin{Verbatim}[commandchars=\\\{\}]
{\color{incolor}In [{\color{incolor}23}]:} \PY{n}{threshold3} \PY{o}{=} \PY{n}{cv2}\PY{o}{.}\PY{n}{adaptiveThreshold}\PY{p}{(}\PY{n}{paper\PYZus{}img\PYZus{}grayscale}\PY{p}{,} 
                                             \PY{l+m+mi}{255}\PY{p}{,} \PY{n}{cv2}\PY{o}{.}\PY{n}{ADAPTIVE\PYZus{}THRESH\PYZus{}MEAN\PYZus{}C}\PY{p}{,}
                                             \PY{n}{cv2}\PY{o}{.}\PY{n}{THRESH\PYZus{}BINARY}\PY{p}{,} \PY{l+m+mi}{201}\PY{p}{,} \PY{l+m+mi}{7}\PY{p}{)}
         \PY{c+c1}{\PYZsh{} the first parameter is the image.}
         \PY{c+c1}{\PYZsh{} the second parameter is the maxVal to which a pixel is converted.}
         \PY{c+c1}{\PYZsh{} the third paramter is the type of the adaptive thresholding.}
         \PY{c+c1}{\PYZsh{} the fourth paramter is the type of the simple thresholding.}
         \PY{c+c1}{\PYZsh{} the fifth type is the selected region size. more bigger this value more adaptive thresholding is closer to simple thresholding.}
         \PY{c+c1}{\PYZsh{} the sixth parameter is just a constant C which is subtracted from the mean or weighted mean calculated.}
         \PY{n}{plt}\PY{o}{.}\PY{n}{imshow}\PY{p}{(}\PY{n}{threshold3}\PY{p}{,} \PY{n}{cmap}\PY{o}{=}\PY{l+s+s1}{\PYZsq{}}\PY{l+s+s1}{gray}\PY{l+s+s1}{\PYZsq{}}\PY{p}{)}
\end{Verbatim}


\begin{Verbatim}[commandchars=\\\{\}]
{\color{outcolor}Out[{\color{outcolor}23}]:} <matplotlib.image.AxesImage at 0x7f504d3bc710>
\end{Verbatim}
            
    \begin{center}
    \adjustimage{max size={0.9\linewidth}{0.9\paperheight}}{output_42_1.png}
    \end{center}
    { \hspace*{\fill} \\}
    
    By looking at this image, you may notice why thresholding is useful and
why I said that CamScanner app is based on this idea.

    \subsection{1.6 Image Arithmetics}\label{image-arithmetics}

    To understand this part, we need two images. For that purpose, I am
going to upload another image. \textbf{Make sure that both images are of
identical size}

    \begin{Verbatim}[commandchars=\\\{\}]
{\color{incolor}In [{\color{incolor}24}]:} \PY{n}{clrd\PYZus{}img}\PY{o}{.}\PY{n}{shape}
\end{Verbatim}


\begin{Verbatim}[commandchars=\\\{\}]
{\color{outcolor}Out[{\color{outcolor}24}]:} (174, 290, 3)
\end{Verbatim}
            
    \begin{Verbatim}[commandchars=\\\{\}]
{\color{incolor}In [{\color{incolor}25}]:} \PY{n}{clrd\PYZus{}img2} \PY{o}{=} \PY{n}{cv2}\PY{o}{.}\PY{n}{imread}\PY{p}{(}\PY{l+s+s1}{\PYZsq{}}\PY{l+s+s1}{colorful\PYZus{}image2.jpeg}\PY{l+s+s1}{\PYZsq{}}\PY{p}{)}
         \PY{n}{plt}\PY{o}{.}\PY{n}{imshow}\PY{p}{(}\PY{n}{clrd\PYZus{}img2}\PY{p}{)}
\end{Verbatim}


\begin{Verbatim}[commandchars=\\\{\}]
{\color{outcolor}Out[{\color{outcolor}25}]:} <matplotlib.image.AxesImage at 0x7f5046c2d128>
\end{Verbatim}
            
    \begin{center}
    \adjustimage{max size={0.9\linewidth}{0.9\paperheight}}{output_47_1.png}
    \end{center}
    { \hspace*{\fill} \\}
    
    The logo, basically, is without background, you may preview it using
opencv to double check. \textbf{However, cv2 adds a black background to
the image!!} This is the original image:
\href{https://upload.wikimedia.org/wikipedia/commons/6/61/King_Fahd_University_of_Petroleum_\%26_Minerals.png}{kfupm
logo}

    \begin{Verbatim}[commandchars=\\\{\}]
{\color{incolor}In [{\color{incolor}27}]:} \PY{n}{cv2}\PY{o}{.}\PY{n}{imshow}\PY{p}{(}\PY{l+s+s1}{\PYZsq{}}\PY{l+s+s1}{kfupm logo}\PY{l+s+s1}{\PYZsq{}}\PY{p}{,}\PY{n}{kfupm\PYZus{}logo}\PY{p}{)}
         \PY{n}{cv2}\PY{o}{.}\PY{n}{waitKey}\PY{p}{(}\PY{l+m+mi}{0}\PY{p}{)}
         \PY{n}{cv2}\PY{o}{.}\PY{n}{destroyAllWindows}\PY{p}{(}\PY{p}{)}
\end{Verbatim}


    \subsubsection{1.6.1 Image Addition}\label{image-addition}

    \paragraph{1.6.1.1 absolute addition}\label{absolute-addition}

    We can interpret both images using only simple addition of images!!
Images are implemented using numpy arrays. Therefore, you can apply this
operation just as any two numpy arrays.

what do you think the image will look like?? See the it below:

    \begin{Verbatim}[commandchars=\\\{\}]
{\color{incolor}In [{\color{incolor}31}]:} \PY{n}{two\PYZus{}images} \PY{o}{=} \PY{n}{clrd\PYZus{}img}\PY{o}{+}\PY{n}{clrd\PYZus{}img2}
         \PY{n}{plt}\PY{o}{.}\PY{n}{imshow}\PY{p}{(}\PY{n}{two\PYZus{}images}\PY{p}{)}
\end{Verbatim}


\begin{Verbatim}[commandchars=\\\{\}]
{\color{outcolor}Out[{\color{outcolor}31}]:} <matplotlib.image.AxesImage at 0x7f504d41fe48>
\end{Verbatim}
            
    \begin{center}
    \adjustimage{max size={0.9\linewidth}{0.9\paperheight}}{output_53_1.png}
    \end{center}
    { \hspace*{\fill} \\}
    
    \textbf{What a bad interpretaion :)) But, What happened??} This image is
the result of adding all pexils of both images!! The operation is in mod
256. That is, c = a+b (mod 256) where c is a pexil in the new resuted
image a,b are pexils in the images to add.

This resulting image suggests that we shall think of another way if we
want to merge two images. It may work sometimes but may give horrible
results just as what we get in our case!!

    Another version of add is provided with opencv itself. follow the code
below:

    \begin{Verbatim}[commandchars=\\\{\}]
{\color{incolor}In [{\color{incolor}32}]:} \PY{n}{two\PYZus{}images\PYZus{}cv} \PY{o}{=} \PY{n}{cv2}\PY{o}{.}\PY{n}{add}\PY{p}{(}\PY{n}{clrd\PYZus{}img}\PY{p}{,} \PY{n}{clrd\PYZus{}img2}\PY{p}{)}
         \PY{n}{plt}\PY{o}{.}\PY{n}{imshow}\PY{p}{(}\PY{n}{two\PYZus{}images\PYZus{}cv}\PY{p}{)}
\end{Verbatim}


\begin{Verbatim}[commandchars=\\\{\}]
{\color{outcolor}Out[{\color{outcolor}32}]:} <matplotlib.image.AxesImage at 0x7f5046e9e860>
\end{Verbatim}
            
    \begin{center}
    \adjustimage{max size={0.9\linewidth}{0.9\paperheight}}{output_56_1.png}
    \end{center}
    { \hspace*{\fill} \\}
    
    This is a better combination though. Nevertheless, what really happens??
Okay, this image is also the result of adding all pexils of both
images!! But if the pexils summatoin is greater than 255, the value of
the new pexil is 255.

For example, if a = 255, b = 10, which are pexils in the added images,
then pexil c, which is in the resulted image is only 255!!

    \paragraph{1.6.1.2 wieghted addition}\label{wieghted-addition}

    OpenCV still provides another alternative if the previous solutions did
not match developers needs. This times, you can combine images based on
a predetermined weight. The code is below:

    \begin{Verbatim}[commandchars=\\\{\}]
{\color{incolor}In [{\color{incolor}33}]:} \PY{n}{two\PYZus{}images\PYZus{}weighted} \PY{o}{=} \PY{n}{cv2}\PY{o}{.}\PY{n}{addWeighted}\PY{p}{(}\PY{n}{clrd\PYZus{}img}\PY{p}{,} \PY{l+m+mf}{0.45}\PY{p}{,} \PY{n}{clrd\PYZus{}img2}\PY{p}{,} \PY{l+m+mf}{0.55}\PY{p}{,} \PY{l+m+mi}{0}\PY{p}{)}
         \PY{n}{plt}\PY{o}{.}\PY{n}{imshow}\PY{p}{(}\PY{n}{two\PYZus{}images\PYZus{}weighted}\PY{p}{)}
\end{Verbatim}


\begin{Verbatim}[commandchars=\\\{\}]
{\color{outcolor}Out[{\color{outcolor}33}]:} <matplotlib.image.AxesImage at 0x7f5046a4d240>
\end{Verbatim}
            
    \begin{center}
    \adjustimage{max size={0.9\linewidth}{0.9\paperheight}}{output_60_1.png}
    \end{center}
    { \hspace*{\fill} \\}
    
    \emph{As a Conclusion}, there are three ways, up to what I know, to
combine images, a developer may choose the best implementation fitting
his needs.


    % Add a bibliography block to the postdoc
    
    
    
    \end{document}
